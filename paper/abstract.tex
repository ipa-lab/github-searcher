Mining software repositories on GitHub has been enabled by datasets like GHTorrent and GitHub's public contents dataset on Google's BigQuery platform.
However, both of these datasets suffer from limitations that make them unsuitable for certain use cases: 
GHTorrent only provides meta-data and omits source code entirely, 
while the BigQuery dataset is incomplete, as it only indexes repositories with an explicit open source license.
The ground truth to mining code is still the GitHub Code Search API.
It allows users to retrieve file contents based on a simple search query.
However, it is heavily rate-limited and only returns up to 1000 results per search.
On the surface, this makes representative sampling, let alone exhaustive mining, impossible.
We developed a tool, github-searcher, that implements a stratified sampling technique to get around this technical limitation.
By repeatedly searching with the same search term but different incremental conditioning on non-overlapping file size ranges, we can reach a good approximative sample of the overall population.
We present a case study in which we sample the search term 'alias' within Shell files to demonstrate the effectiveness of our approach.